\begin{tabular}{|p{\textwidth}|}
\hline
\textbf{Klasa: CollisionShape}\\ \hline
\emph{Nazwa: } getTransform(Vector3 position, Quaternion rotation) :void\\
\emph{Opis: } Metoda pobiera transformację obiektu: wektor pozycji oraz
kwaternion obrotu i zwraca je przez parametry.\\
\hline
\emph{Nazwa: } getTranslation() :Vector3\\
\emph{Opis: } Metoda pobiera wektor pozycji.\\
\hline
\emph{Nazwa: } getRotation() :Quaternion\\
\emph{Opis: } Metoda pobiera kwaternion obrotu.\\
\hline
\emph{Nazwa: } setTranslation(Vector3 position) :void\\
\emph{Opis: } Metoda ustawia centrum obiektu w danym punkcie.\\
\hline
\emph{Nazwa: } getRotation(Quaternion rotation) :void\\
\emph{Opis: } Metoda ustawia obrót obiektu względem podanego kwaternionu.\\
\hline
\emph{Nazwa: } applyCentralForce(Vector3 force): void\\
\emph{Opis: } Metoda działa na centrum obiektu siłą o wartości podanej w
parametrze.\\
\hline
\emph{Nazwa: } applyTorque(Vector3 torque): void\\
\emph{Opis: } Metoda działa na obiekt z momentem siły o wartości podanej w
parametrze.\\
\hline
\emph{Nazwa: } render(GL10 gl): void\\
\emph{Opis: } Metoda renderuje bryłę sztywną w oparciu o podany podczas
tworzenia obiektu kształt.\\
\hline
\emph{Nazwa: } setRestitution(float restitution): void\\
\emph{Opis: } Metoda ustawia wartość współczynnika sprężystości bryły.\\
\hline
\emph{Nazwa: } setFriction(float friction): void\\
\emph{Opis: } Metoda ustawia wartość współczynnika tarcia bryły.\\
\hline
\emph{Nazwa: } setLinearVelocity(Vector3 velocity): void\\
\emph{Opis: } Metoda ustawia wartość liniowej prędkości bryły.\\
\hline
\emph{Nazwa: } setAngularVelocity(Vector3 velocity): void\\
\emph{Opis: } Metoda ustawia wartość prędkości kątowej bryły.\\
\hline
\emph{Nazwa: } getLinearVelocity(): Vector3\\
\emph{Opis: } Metoda pobiera wartość liniowej prędkości bryły.\\
\hline
\emph{Nazwa: } getAngularVelocity(): Vector3\\
\emph{Opis: } Metoda pobiera wartość prędkości kątowej bryły.\\
\hline
\end{tabular}