\begin{tabular}{|p{\textwidth}|}
\hline
\textbf{Klasa: Spring}\\
\hline
\emph{Nazwa: } konstruktor(CollisionShape shapeA, CollsionShape shapeB, Vector3
frameInA, Vector3 frameInB)\\
\emph{Opis: } Tworzona sprężyna łączy ze sobą dwa obiekty \emph{shapeA} -
oznaczone na rysunku przez ciało \textbf{A} i \emph{shapeB} - oznaczone przez
ciało \textbf{B}. Parametr \emph{frameInA} określa wektor przesunięcia ramki
ograniczającej swobodę zawieszonego ciała \textbf{B} oznaczony na rysunku przez
wektor \textbf{C}. Parametr \emph{frameInB} odpowiada za wektor przesunięcia
ciała \textbf{B} względem ograniczenia osi obrotu(oznaczonego przez łuk
\textbf{F}).\\
\hline
\emph{Nazwa: } enable(Dof index, boolean enable) :void\\
\emph{Opis: } Metoda odblokowuje dany stopień swobody.\\
\hline
\emph{Nazwa: } setDamping(Dof index, float damping): void\\
\emph{Opis: } Metoda ustawia wartość współczynnika tłumienia dla danego
stopnia swobody.\\
\hline
\emph{Nazwa: } setStiffness(Dof index, float stiffness) :void\\
\emph{Opis: } Metoda ustawia wartość współczynnika sztywności dla danego
stopnia swobody.\\
\hline
\emph{Nazwa: } setupDof(Dof index, float stiffness, float damping) :void\\
\emph{Opis: } Metoda pozwalająca na równoczesne ustawienie współczynnika
sztywności i tłumienia dla danego stopnia swobody.\\
\hline
\emph{Nazwa: } setLinerUpperLimit(Vector3 vec) :void i
setLinerLowerLimit(Vector3 vec) :void\\
\emph{Opis: } Wektory \emph{vec} określają kolejno dolne i górne wartości ramki
ograniczającej swobodę ciała \textbf{B}. Środek ramki znajduje się w pozycji
ciała \textbf{A} i jest przesunięty o wektor \emph{frameInA} podany w
konstruktorze. Na rysunku pozycję ograniczenia ramki oznaczono przez punkty
\textbf{D} i \textbf{E}.\\
\hline
\emph{Nazwa: } setAngularUpperLimit(Vector3 vec) :void i
setAngularLowerLimit(Vector3 vec) :void\\
\emph{Opis: } Metody określają odpowiednio górny i dolny limit rotacji ciała
\textbf{B}. Na rysunku ograniczenie jest oznaczone przez łuk \textbf{F}. Ciało
\textbf{B} może być przesunięte względem środka ograniczenia obrotu o wektor
\emph{frameInB} podany w konstruktorze klasy Spring.\\
\hline
\end{tabular}