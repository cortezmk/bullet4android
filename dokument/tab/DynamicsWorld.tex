\begin{tabular}{|p{\textwidth}|}
\hline
\textbf{Klasa: DynamicsWorld}\\ \hline
\emph{Nazwa: } createShape(Mesh mesh, Vector3 position, float mass)
:CollisionShape\\
\emph{Opis: } Metoda tworzy nowy obiekt \emph{CollisionShape}(bryłę sztywną) i
dodaje go do środowiska fizycznego. Kształt i rozmiary obiektu zależą od
parametru \emph{mesh}.\\
\hline
\emph{Nazwa: } setGravity(Vector3 gravity) :void\\
\emph{Opis: } Metoda ustawia wektor przyspieszenia grawitacyjnego dla wszystkich
obiektów w środowisku.\\
\hline
\emph{Nazwa: } stepSimulation(int timeStep) :void\\
\emph{Opis: } Metoda aktualizuje symulację o podany czas w milisekundach.\\
\hline
\emph{Nazwa: } pickObject(Vector3 rayFrom, Vector3 rayTo) :void\\
\emph{Opis: } Metoda używana przez akcję ``przeciągnij i upuść" opisaną w
sekcji \ref{sec:dragAndDrop}.\\
\hline
\emph{Nazwa: } dragObject(Vector3 rayFrom, Vector3 rayTo) :void\\
\emph{Opis: } Metoda używana przez akcję ``przeciągnij i upuść" opisaną w
sekcji \ref{sec:dragAndDrop}.\\
\hline
\emph{Nazwa: } dropObject(Vector3 rayFrom, Vector3 rayTo) :void\\
\emph{Opis: } Metoda używana przez akcję ``przeciągnij i upuść" opisaną w
sekcji \ref{sec:dragAndDrop}.\\
\hline
\emph{Nazwa: } drawDebug(GL10 gl) :void\\
\emph{Opis: } Metoda używana podczas renderowania linii pomocniczych. Opisana
jest w sekcji \ref{sec:debugDrawer}.\\
\hline
\end{tabular}